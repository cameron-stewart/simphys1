\documentclass[11pt,a4paper]{scrartcl}

% Pakete
\usepackage[english]{babel}
\usepackage[UKenglish]{isodate}
\usepackage{xcolor}
\usepackage{graphicx}
\usepackage{amsmath}
\usepackage{amssymb}
\usepackage{nicefrac}
\usepackage[utf8]{inputenc}
\usepackage{siunitx}
\sisetup{
output-decimal-marker={.},
exponent-product=\cdot }
\usepackage{esvect}
\usepackage{eqnarray}
\usepackage{placeins}
\usepackage{scrpage2}
\usepackage{nameref}
\usepackage{upgreek}
\usepackage{caption}
\usepackage{subcaption}
\usepackage{bm}
\usepackage{mwe}
\usepackage{tcolorbox}
\usepackage{listings}
\usepackage{xstring}
\usepackage{stringstrings}
\usepackage{floatflt}
\usepackage{pgfplots}
\usepackage{tikz}
%\usepackage{physics}


% Custom colors
\definecolor{deepblue}{rgb}{0,0,0.5}
\definecolor{deepred}{rgb}{0.6,0,0}
\definecolor{deepgreen}{rgb}{0,0.5,0}
\DeclareFixedFont{\ttb}{T1}{txtt}{bx}{n}{12} % for bold
\DeclareFixedFont{\ttm}{T1}{txtt}{m}{n}{12}  % for normal

% tikz
\usetikzlibrary{arrows}

% caption setup
\captionsetup[subfigure]{labelformat=simple, labelsep=colon}
\renewcaptionname{english}{\figurename}{Fig.}
\renewcommand{\thesubfigure}{\arabic{figure}.\arabic{subfigure}}
\renewcommand{\thesubtable}{\arabic{table}.\arabic{subtable}}

% Listning Einstellungen
\lstloadlanguages{python}       % Default highlighting set to "python"
\DeclareCaptionFont{white}{\color{white}}
\DeclareCaptionFormat{listing}{%
  \parbox{0.99\textwidth}{\colorbox{gray}{\parbox{0.99\textwidth}{#1#2#3}}\vskip+5pt}}
\captionsetup[lstlisting]{format=listing, labelfont=white, textfont=white}
\lstset{frame=lrb,xleftmargin=\fboxsep,xrightmargin=-\fboxsep}
\pagestyle{empty}
\lstset{escapeinside={<@}{@>}}
\lstdefinestyle{MyPythonStyle}{
		language=Python,
		numbers=left,
		breaklines=true,
		basicstyle=\ttm,
		otherkeywords={self},             % Add keywords here
		keywordstyle=\ttb\color{deepblue},
		emph={MyClass,__init__},          % Custom highlighting
		emphstyle=\ttb\color{deepred},    % Custom highlighting style
		stringstyle=\color{deepgreen},
		frame=tb,                         % Any extra options here
		showstringspaces=false            % 
		}
\lstdefinestyle{MyCStyle}{
		language=C,
		numbers=left,
		tabsize=4,
		breaklines=true,
		basicstyle=\ttm,
		otherkeywords={self},             % Add keywords here
		keywordstyle=\ttb\color{deepblue},
		emph={MyClass,__init__},          % Custom highlighting
		emphstyle=\ttb\color{deepred},    % Custom highlighting style
		stringstyle=\color{deepgreen},
		frame=tb,                         % Any extra options here
		showstringspaces=false            % 
		}
\renewcommand{\lstlistingname}{Code block}
\renewcommand{\lstlistlistingname}{List of \lstlistingname s}

% Griechische Buchstaben vereinheitlichen
\renewcommand{\alpha}{\upalpha}
\renewcommand{\beta}{\upbeta}
\renewcommand{\gamma}{\upgamma}
\renewcommand{\delta}{\updelta}
\newcommand{\w}{\omega}
\newcommand{\la}{\lambda}

% Eigene mathematische Kommandos
\newcommand{\dd}{\text{d}} 							% Differential
\newcommand{\p}{\partial} 								% Partielles Differential
\newcommand{\D}{\Delta} 								% Fehler / Laplace
\newcommand{\order}[1]{\mathcal{O}\left( #1 \right)}
\newcommand{\abs}[1]{\left| #1\right|} 					% Betrag
\newcommand{\Max}[1]{\max \left\lbrace #1\right\rbrace} % max{}
\newcommand{\Min}[1]{\min \left\lbrace #1\right\rbrace} % min{}
\newcommand{\diff}[2]{\frac{\text{d} #1}{\text{d} #2}}	% Ableitung
\newcommand{\pdiff}[2]{\frac{\partial #1}{\partial #2}} % Partielle Ableitung
\newcommand{\errprop}[2]{\left| \frac{\partial #1}{\partial #2}\right| \cdot \Delta #2}

% Klammern
\newcommand{\lk}{\left\langle}
\newcommand{\rk}{\right\rangle}
\newcommand{\lb}{\left\lbrace}
\newcommand{\rb}{\right\rbrace}
\newcommand{\lc}{\left(}
\newcommand{\rc}{\right)}
												% Fehlerfortpflanzung
\newcommand{\rel}[1]{\frac{\Delta #1}{#1}}				% Relativer Fehler

% Eigene trigonometrische Funktionen
\newcommand{\Exp}[1]{\text{exp}\left( #1 \right)}		% exp()
\newcommand{\Ln}[1]{\text{ln}\left( #1 \right)}			% ln()
\newcommand{\Log}[1]{\text{log}\left( #1 \right)}		% log()
\newcommand{\Sin}[1]{\text{sin}\left( #1 \right)}     	% sin()
\newcommand{\Cos}[1]{\text{cos}\left( #1 \right)}		% cos()
\newcommand{\Sinz}[1]{\text{sin}^2\left( #1 \right)}	% sin^2()
\newcommand{\Cosz}[1]{\text{cos}^2\left( #1 \right)}	% cos^2()
\newcommand{\Tan}[1]{\text{tan}\left( #1 \right)}		% tan()
\newcommand{\Asin}[1]{\text{asin}\left( #1 \right)}		% asin()
\newcommand{\Acos}[1]{\text{acos}\left( #1 \right)}		% acos()
\newcommand{\Atan}[1]{\text{atan} \left( #1 \right)}	% atan()

% Eigene Vektor Kommandos
\newcommand{\tovec}[2]{\begin{pmatrix}#1\\ #2\end{pmatrix}}	% 2D-Vektor
\newcommand{\trvec}[3]{\begin{pmatrix}#1\\ #2\\ #3\end{pmatrix}}	% 3D-Vektor	
\newcommand{\ovec}[1]{\boldsymbol{#1}}

% Eigene Referenz-Kommandos
\newcommand{\eref}[1]{(\ref{#1})}						% Gleichungen
\newcommand{\sref}[2]{\subref{#2}}						% Unterabbildungen
\newcommand{\kref}[1]{\ref{#1} \glqq\nameref{#1}\grqq}  % Kapitel
\newcommand{\lref}[1]{$[#1]$}							% Quellen: \newcommand{\lit}{1} => \lref{\lit}

% listing commandos
\newcommand{\listfile}[7][MyPythonStyle]{
\lstinputlisting[linerange={#4-#5}, firstnumber=#4, caption={#6} \hfill script:  #3, label=#7, style=#1]{#2}}
% arguments:
% \listfile[style]{location/filename}{filename}{firstline}{lastline}{title}{label}
% Imports code from a file 
% You will have to escape the filename and the title
% style is an optional argument.

\newcommand{\ls}[1]{\lstinline@#1@}

% using \lstinline@code@ for code in line
% works with every sign instead of @

% commands for this task
\newcommand{\ua}{AU}
\newcommand{\vecx}[1]{\ovec{x}^{(#1)}}
\newcommand{\vecv}[1]{\ovec{v}^{(#1)}}
\newcommand{\veca}[1]{\ovec{a}^{(#1)}}
\newcommand{\vecF}[1]{\ovec{F}^{(#1)}}
\newcommand{\vecr}[1]{\ovec{r}^{(#1)}}
\newcommand{\nx}[1]{x^{(#1)}}
\newcommand{\nv}[1]{v^{(#1)}}
\newcommand{\na}[1]{a^{(#1)}}
\newcommand{\nm}[1]{m^{(#1)}}
\newcommand{\nF}[1]{F^{(#1)}}
\newcommand{\nr}[1]{r^{(#1)}}
\newcommand{\Ekin}{E_\text{kin}}
\newcommand{\Ekino}{E_{\text{kin},0}}
\newcommand{\fr}{f_\text{re}}
\newcommand{\fij}{\ovec{f}_{ij}}
\newcommand{\rij}{\ovec{r}_{ij}}
\newcommand{\fji}{\ovec{f}_{ji}}
\newcommand{\rji}{\ovec{r}_{ji}}
\newcommand{\rj}{\ovec{r}_{j}}
\newcommand{\ri}{\ovec{r}_{i}}



% Kopf-/Fusszeile
\pagestyle{scrheadings}
\clearscrheadfoot
\chead{Simulation Methods in Physics I}
\ihead{Worksheet 3}
\ohead{\today}
\ofoot{\pagemark}
\ifoot{Michael Marquardt, Cameron Stewart}

\begin{document}

% Titelseite
\begin{titlepage}

\ \\ \ \\ \ \\

\center\textbf{
\begin{large}
Simulation Methods in Physics I
\end{large} \\ \ \\
\begin{Large}
Worksheet 5: Monte-Carlo
\end{Large}}   \\ \ \\
\ \\ \ \\

\begin{tabular}{lll}
Students: &Michael Marquardt &Cameron Stewart\\ 
matriculation numbers: &3122118 &3216338\\
\end{tabular}

\end{titlepage}

% Inhaltsverzeichnis

% -------------------------------------- Begin Of Document ----------------------------------------

\section{Simple Sampling - Integration}
In this worksheet we will use the Monte Carlo technique to integrate a function and to look at the Ising 
spin model. The Monte Carlo method is a statistical technique for computing expectation values that are 
given by
\begin{equation}
\langle A \rangle = \frac{\int_\Phi A(\phi) P(\phi) d\phi}{\int P(\phi) d\phi}
\label{eq:exp}
\end{equation}
where $A$ is an observable, $P$ is the probability of finding the system in a certain state, $\phi$ is a state of the system and $\Phi$ is the set of all states of the system.

The Monte Carlo simply replaces these integrals with finite sums over random states of the system as
\begin{equation}
\langle A \rangle = \frac{\sum_N A(\phi) P(\phi) d\phi}{\sum_N P(\phi) d\phi}.
\label{eq:mc}
\end{equation}
If the states $\phi$ are randomly distributed then this is called simple sampling.

We can use the Monte Carlo method to evaluate an integral if we set $\phi = x$ and let $P(x)$ be constant. Equate eqs. \ref{eq:exp} and \ref{eq:mc} and solve for the integral leaves us with
\begin{equation}
\int_a^b f(x) dx = \frac{b-a}{N} \sum_N f(x_i)
\label{eq:int}
\end{equation}
where we have renamed $A$ to $f$.

We begin this worksheet by using this method to evaluate the integral of the Runge function over the interval $[-5, 5]$. The Runge function is given by
\begin{equation}
f(x) = \frac{1}{1+x^2}
\end{equation}
and its integral is 
\begin{equation}
\int_a^b \frac{1}{1+x^2} dx = \arctan(b)-\arctan(a).
\end{equation}

We begin by writing a simple function in python to compute the Runge function and another to compute its integral.
\listfile{../src/mon_car.py}{src/mon\_car.py}{3}{7}{Runge Function}{runge}
\begin{figure}[ht]
	\centering
	\includegraphics[width=0.7\textwidth]{../fig/rungeplot.png}
	\caption{Plot of the Runge function from $-5$ to $5$.	}
	\label{fig:runge}
\end{figure}
The Runge function is plotted in fig. \ref{fig:runge}

Now we will use a simple sampling Monte Carlo technique to evaluate the integral. Equation \ref{eq:int} is implemented as:
\listfile{../src/mon_car.py}{src/mon\_car.py}{9}{12}{Simple Sampling}{simple}
This allows us to set a function, upper and lower bound, and number of samples as parameters. The measured value of the integral and the standard error of the mean are returned. 

We now write a program to use our Monte Carlo algorithm to compute this integral for a number of samples $N = 2^i$ with $2 \leq i \leq 20$. 
\listfile{../src/integration.py}{src/integration.py}{5}{13}{Simple Sampling}{simpsamp}
\begin{figure}[ht]
	\centering
	\includegraphics[width=0.7\textwidth]{../fig/simple_err.png}
	\caption{Plot of the statistical vs actual error in our simple sampling evaluation of the integral of the Runge function for various number of samples.	}
	\label{fig:simple}
\end{figure}
Figure \ref{fig:simple} shows the statistical vs. actual error in our calculation of this integral for our range of samples. We used a log-log plot for readability.

\section{Importance Sampling - Metropolis-Hastings Algorithm}



\end{document}

