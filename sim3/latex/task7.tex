\subsection{Warming up the System}

When warming up the system you want to state a initial maximum force.
Therefor the command line option \ls{--warm} can be used (code block \ref{comsim}).
The option \ls{--cwarm} works analougus to \ls{--ctstat}.\\

The first thing needed is the function c\_force\_capping() which is implemented in C. 
The reason for this is because you have to do many if requests and calculations for each particle.

\listfile[MyCstyle]{../src/c_lj.cpp}{src/c\_lj.cpp}{288}{305}{Force capping}{7forcap}

The function in code block \ref{7forcap} limits the forces to a given maximum force fcap, but conserves the direction if the forces. 
Although this process is not physically correct.
Furthermore the function tests if no force is capped any more. 
When this is the case it sets fcap to zero.\\

The function is called in the vv\_step() function in ljsim.py.

\listfile{../src/ljsim.py}{src/ljsim.py}{164}{168}{Limiting forces}{7limfor}

As you can see, the function will not be activated any more if fcap = args.warm is set to zero.\\

At least the fcap should be increased by ten percent every time the measurements are done. 
This is done in code block \ref{7inc}.

\listfile{../src/ljsim.py}{src/ljsim.py}{213}{215}{Increase maximum force}{7inc}

But all the capping is useless without random initial conditions. 
The velocity is already random, so only the initial position has to be changed. 
You can do this with a similar command like in \ref{7rand}.

\listfile{../src/ljsim.py}{src/ljsim.py}{98}{101}{Random initial position}{7rand}
