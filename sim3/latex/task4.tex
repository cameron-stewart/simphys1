\section{Calculating Temperature and Pressure}

\subsection{Expanding Energy Calculation}

First the energy calculation should be changed in order to also get the kinetic and potential energy.
This is done by changing the compute\_energy() function call and the parameter Es in ljsim.py (code block \ref{4energy}).

\listfile{../src/ljsim.py}{src/ljsim.py}{179}{188}{$E_\text{kin}$ + $E_\text{pot}$}{4energy}

\subsection{Calculating the Temperature}

The temperature of the system can be calculated from the kinetic Energy E\_kin like in code block \ref{4energy} shown. This is a single scalar calculation and therefore can be done in python.

\subsection{Calcating Pressure}

\subsubsection*{Derivation of the pressure}

The virial of a system is defined as:
\begin{align}
G
	&=\sum_i^N\ovec{p}_i\cdot\ovec{r}_i
	\label{4p1}
\end{align}

In the case
\begin{align}
0
	&=\left\langle\diff{G}{t}\right\rangle 
	=\left\langle\sum_i^N\frac{\ovec{p}_i^2}{m_i}\right\rangle+\sum_i^N\left\langle\ovec{F}_i\cdot\ovec{r}_i\right\rangle
	\label{4p2}
\end{align}

the following equation can be derived:
\begin{align}
-\sum_i^N\left\langle\ovec{F}_i\cdot\ovec{r}_i\right\rangle
	&=2\left\langle\sum_i^N\frac{\ovec{p}_i^2}{2m_i}\right\rangle
	=2\left\langle E_\text{kin}\right\rangle
	=3Nk_BT
	\stackrel{\text{id. gas}}{=}3PV
	\label{4p3}
\end{align}

For an pair interaction (like in the simulation) let the forces $\fij$ and the vectors $\rij$ be defined as follows:
\begin{align}
\rij
	&=\rj - \ri
	\label{4rij}\\
\fij 
	&= f_\text{lj}(\rij^2)\cdot\frac{\rij}{\abs{\rij}}
	\label{4fij}
\end{align}

With this knowldege we can derive the pressure of the system:
\begin{align}
P
	&=P_\text{id. gas} + P_\text{interaction}
	\label{4p4}\\
	&=\frac{Nk_BT}{V} + \frac{1}{3V}\sum_i^N\left\langle\ovec{F}_{i,\text{interaction}}\cdot\ovec{r}_i\right\rangle
	\label{4p5}\\
	&=\frac{1}{3V}\left[\left\langle\sum_i^N\frac{\ovec{p}_i^2}{2m_i}\right\rangle + \left\langle\sum_{i,j\neq i}^N-\fij\cdot\ovec{r}_i\right\rangle\right]
	\label{4p6}\\
	&=\frac{1}{3V}\left[\left\langle\sum_i^N\frac{\ovec{p}_i^2}{2m_i}\right\rangle + \left\langle\sum_{i,j>i}^N\fij\cdot\rj-\fij\cdot\ri\right\rangle\right]
	\label{4p7}\\
	&=\frac{1}{3V}\left[\left\langle\sum_i^N\frac{\ovec{p}_i^2}{2m_i}\right\rangle + \left\langle\sum_{i,j>i}^N\fij\cdot\rij\right\rangle\right]
	\label{4p8}\\
	&=\frac{1}{3V}\left[ 2\left\langle E_\text{kin}\right\rangle + \left\langle\sum_{i,j>i}^N\fij\cdot\rij\right\rangle\right]
	\label{4p9}
\end{align}

\subsection*{Pressure in Cython}

Becaus of the vectorial calculations for each particle pair the calculation has to be done in C if it should be fast. 
Therefor the function c\_compute\_pressure() is written in c\_lj.cpp.

\listfile[MyCstyle]{../src/c_lj.cpp}{src/c\_lj.cpp}{253}{279}{$E_\text{kin}$ + $E_\text{pot}$}{4pressure}

As you can see the first part is the same like in c\_compute\_forces(), but the last part differs.
In order to save calculation time the function takes the kinetic energy, which is already calculated, as an argument.